\documentclass[12pt, 14paper]{article}

\usepackage{amsthm}
\usepackage{amsmath, amsfonts}
\usepackage{mathtools,array,booktabs,mathabx}
\usepackage{natbib}
\usepackage[inline]{enumitem}
\usepackage{forest}
\usepackage{diagbox}
\usepackage{bussproofs}
\usepackage[tableaux]{prooftrees}
\EnableBpAbbreviations
%\usepackage[left=.5in,right=.5in,top=0.5in,bottom=.5in]{geometry}
\usepackage{qtree}
\usepackage{pdflscape}
\usepackage{multicol}
\usepackage{xepersian}

\settextfont[Scale=1.3]{IRBadr}
\linespread{1.2}

\usepackage{versions}
\excludeversion{ans}



\renewcommand\beginmarkversion{\\\textbf{پاسخ: }}
\renewcommand\endmarkversion{$\qed$}

\forestset{line numbering = false, just sep = 0em, check with = {:}}

\markversion{ans}

\title{سری اول تمرینات درس مبانی منطق}
\author{}
\date{}

\begin{document}

\maketitle

\vspace{-2.5cm}

\begin{center}آخرین زمان تحویل: ۲۴ اسفندماه، ساعت ۲۳:۵۹\end{center}

\vspace{0.5cm}

\begin{enumerate}

\item
ثابت کنید مجموعه‌ی تمام عبارات زبان منطق گزاره‌ها با مجموعه‌ی اتم‌های شمارا، شمارا است.
\begin{ans}
  مطابق قضیه‌ی کانتور-شرودر-برنشتاین کافی است تابعی یک‌به‌یک از $\mathbb{N}$ به توی مجموعه‌ی فرمول‌ها و تابعی یک‌به‌یک از مجموعه‌ی فرمول‌ها به توی $\mathbb{N}$ معرفی کنیم. برای تابع اول این تابع را در نظر بگیرید:
  $$
  f(n)=p_n
  $$
  یک‌به‌یک بودن  این تابع واضح است. برای تابع دوم این تابع را در نظر بگیرید (چنین تابعی را یک \emph{عددگذاری گودلی} می‌نامیم):
  $$
  g(A)=
  \begin{cases}
  2\cdot 3^{n+1} & \text{if}~~A = p_n\\
  2^2\cdot 3^{g(A_1)} & \text{if}~~A=(\neg A_1)\\
  2^3\cdot 3^{g(A_1)}\cdot 5^{g(A_2)} & \text{if}~~A=(A_1\wedge A_2) \\
  2^4\cdot 3^{g(A_1)}\cdot 5^{g(A_2)} & \text{if}~~A=(A_1\vee A_2) \\
  2^5\cdot 3^{g(A_1)}\cdot 5^{g(A_2)} & \text{if}~~A=(A_1\rightarrow A_2)
  \end{cases}
  $$
  با استفاده از قضیه‌ی اساسی حساب می‌توان نشان داد تنها در صورتی $g(A)=g(B)$ که $A=B$ و بنابراین $g$ یک‌به‌یک است.  
\end{ans}
\item
برای هر یک از رشته‌های زیر یا درخت تجزیه‌ی آن را رسم کنید یا از طریق تلاش برای رسم درخت تجزیه نشان بدهید آن رشته یک فرمول نیست:
\begin{enumerate}
\item $((((\neg p_1)\rightarrow\bot)\vee p_1)\wedge p_2)$
\item $(((\neg (p_0\vee p_1))\wedge(p_2\rightarrow p_3)))\rightarrow (p_3\wedge p_4))$
\item[(پ)] $(p_1\wedge\rightarrow\neg(p_2\vee p_0))$
\end{enumerate}\quad\vspace{-9mm}
\begin{ans}
  \begin{enumerate}
    \item \quad\\
    \begin{forest}
      [$\wedge$
        [$\vee$
          [$\rightarrow$
            [$\neg$
              [$p_1$]
            ]
            [$\bot$]
          ]
          [$p_1$]
        ]
        [$p_2$]
      ]
    \end{forest}

    \item طبق الگوریتم تولید درخت تجزیه، ابتدا عمق فرمول و عمق رابط‌های منطقی را محاسبه می‌کنیم. اما عمق این عبارت $-1$ است که نشان می‌دهد این عبارت یک فرمول درست‌ساخت نیست.

    \item طبق الگوریتم تولید درخت تجزیه، باید تنها رابط دارای عمق $1$ را در عبارت پیدا کنیم. اما بعد از محاسبه عمق رابط‌ها، مشاهده می‌شود که سه رابط با عمق $1$ وجود دارد.
  \end{enumerate}
\end{ans}

\item
ثابت کنید:
\begin{enumerate}
\item
اگر $c$ تعداد جایگاه‌هایی در فرمول $A$ باشد که رابطی دوتایی در آن قرار گرفته و $s$ تعداد جایگاه‌هایی در $A$ باشد که یک اتم در آن قرار گرفته، داریم $s=c+1$.
\item
اگر $A$ فرمولی باشد که در آن از $\neg$ استفاده نشده است، طول $A$ فرد است.
\end{enumerate}\quad\vspace{-9mm}
\begin{ans}
  \begin{enumerate}
    \item
    حکم را از طریق استقرا ثابت می‌کنیم. اگر $A$ اتم باشد حکم بدیهی است. حال فرض کنید حکم برای فرمول‌هایی با پیچیدگی کمتر از پیچیدگی $A$ ثابت شده است. اگر $A=(\neg A_1)$ و $c_1$ تعداد جایگاه‌های رابط‌های دوتایی در $A_1$ و $s_1$ تعداد جایگاه‌های اتم‌ها در $A_1$ باشد، واضح است که $s=s_1$ و $c=c_1$ و بنابراین $s=c+1$. همچنین اگر $A=(A_1\square A_2)$ (که در آن $\square$ رابطی دوتایی است) و $c_i$ تعداد جایگاه‌های رابط‌های دوتایی در $A_i$ و $s_i$ تعداد جایگاه‌های اتم‌ها در $A_i$ باشد، داریم $c=c_1+c_2+1$ و $s=s_1+s_2$. حال با توجه به اینکه فرض کرده‌ایم $s_i=c_i+1$، به‌سادگی می‌توان نشان داد $s=c+1$.
    
    
    \item
    حکم را از طریق استقرا ثابت می‌کنیم. اگر $A$ اتم باشد حکم بدیهی است. حال فرض کنید حکم برای فرمول‌هایی با پیچیدگی کمتر از پیچیدگی $A$ ثابت شده است. تنها لازم است حالتی را بررسی کنیم که در آن $A=(A_1\square A_2)$ (که در آن $\square$ رابطی دوتایی است) زیرا در صورتی که $A=(\neg A_1)$ فرض استفاده نشدن از نقیض برقرار نیست. حال واضح است که اگر طول $A_1$ و $A_1$ فرد باشد طول $A$ نیز فرد است.
    
  \end{enumerate}
\end{ans}

\item
فرض کنید $\models A\rightarrow B$ و $A$ و $B$ دارای اتم‌های مشترک نیستند. ثابت کنید یا $A$ ارضاناپذیر است یا $B$ توتولوژی است (یا هر دو). توضیح بدهید که آیا شرط اتم مشترک نداشتن برای این حکم ضروری است یا نه.
\begin{ans}
  با استفاده از قضیه‌ی استنتاج می‌توانیم نتیجه بگیریم $A\models B$. بنابراین هر ارزیاب $v$ که $A$ را ارضا کند، $B$ را نیز ارضا می‌کند. برای اثبات حکم کافی است فرض کنیم $A$ ارضاناپذیر نیست و نتیجه بگیریم $B$ توتولوژی است. اگر $A$ ارضاناپذیر نباشد، ارزیاب $v$ای وجود دارد که $A$ را ارضا می‌کند. حال یک ارزیاب دلخواه $u$ را در نظر می‌گیریم و نشان می‌دهیم $B$ را ارضا می‌کند. ارزیاب $v'$ را به شکل زیر تعریف می‌کنیم:

  $$
  v'(p_n)=
  \begin{cases}
  u(p_n) & \text{if}~~p_n\in atoms(B) \\
  v(p_n) & \text{otherwise}
  \end{cases}
  $$

  حال واضح است که $v'$ نیز همانند $v$ فرمول $A$ را ارضا می‌کند زیرا مقدار این دو ارزیاب در اتم‌های موجود در $A$ یکسان است و در جزوه ثابت کرده‌ایم اگر دو ارزیاب مقادیر یکسانی به اتم‌های داخل یک فرمول نسبت بدهند، به فرمول نیز مقدار یکسانی نسبت می‌دهند. بنابراین از آنجا که $v'$ فرمول $A$ را ارضا می‌کند فرمول $B$ را نیز ارضا می‌کند. نیز، از آنجا که مقداری که ارزیاب $v'$ به اتم‌های $B$ نسبت می‌دهد همان مقداری است که ارزیاب $u$ به آن‌ها نسبت می‌دهد، مقداری که این دو ارزیاب به $B$ نسبت می‌دهند یکسان است. بنابراین ارزیاب $u$ نیز $B$ را ارضا می‌کند. با توجه به اینکه ارزیاب $u$ دلخواه است، هر ارزیابی $B$ را ارضا می‌کند و بنابراین $B$ توتولوژی است.

  در صورتی که شرط مشترک نبودن اتم‌ها را حذف کنیم حکم برقرار نیست؛ مثلاً
  $\models p\rightarrow p$
  اما $p$ نه ارضاناپذیر است و نه توتولوژی.
\end{ans}

\item
اثبات یا رد کنید:

\begin{enumerate}
\item
اگر $A$ یک $\{\wedge,\vee,\rightarrow,\leftrightarrow\}$-فرمول باشد، آنگاه $A$ ارضاشدنی است.
\item
اگر $A$ یک $\{\wedge,\vee\}$-فرمول باشد، آنگاه $A$ توتولوژی نیست.
\end{enumerate}\quad\vspace{-9mm}
\begin{ans}
\begin{enumerate}
  \item
  فرض کنید $v$ ارزیابی باشد که به همه‌ی اتم‌ها مقدار $T$ را نسبت می‌دهد. نشان می‌دهیم $v(A)=T$. اگر $A$ اتم باشد حکم واضح است. حال فرض کنید $A$ اتم نیست و حکم درباره‌ی فرمول‌هایی با پیچیدگی کمتر ثابت شده است. در این صورت یا $A=(A_1\square A_2)$ ($\square\in\{\wedge,\vee,\rightarrow,\leftrightarrow\}$) و $v(A_1)=v(A_2)=T$. واضح است که برای هر چهار رابط $v(A)=T$. بنابراین $A$ تحت لااقل یک ارزیاب صادق است و ارضاناپذیر نیست.
  
  \item
  فرض کنید $v$ ارزیابی باشد که به همه‌ی اتم‌ها مقدار $F$ را نسبت می‌دهد. نشان می‌دهیم $v(A)=F$. اگر $A$ اتم باشد حکم واضح است. حال فرض کنید $A$ اتم نیست و حکم درباره‌ی فرمول‌هایی با پیچیدگی کمتر ثابت شده است. در این صورت یا $A=(A_1\wedge A_2)$ یا $A=(A_1\vee A_2)$ و $v(A_1)=v(A_2)=F$. واضح است که در هر دو حالت $v(A)=F$. بنابراین $A$ تحت لااقل یک ارزیاب کاذب است و توتولوژی نیست.
  
  \end{enumerate}
\end{ans}

\item
شخصی غریبه به جزیره‌ای وارد می‌شود که برخی ساکنان آن همیشه دروغ می‌گویند و برخی ساکنان آن همیشه راست. غریبه به دو ساکن جزیره با نام‌های الف و ب می‌رسد و از الف می‌پرسد که «آیا یکی از شما دو نفر دروغگو است؟» الف پاسخی بله/خیر می‌دهد که باعث می‌شود غریبه بتواند راستگو بودن یا دروغگو بودن هر دو نفر را تعیین کند. با نوشتن گزاره‌ی مورد سؤال («آیا یکی از شما دو نفر دروغگو است؟») به زبان صوری و بررسی شرایط صدق آن، مشخص کنید پاسخ الف چه بوده است.
\begin{ans}
  اتم‌
  $p$
  را به «الف دروغگو است.» و اتم
  $q$
  را به «ب دروغگو است.» تعبیر کنید. حال می‌توان سوال «آیا یکی از شما دو نفر دروغگو است؟» را به شکل این سوال ترجمه کرد که آیا گزاره‌ی $p \vee q$ صادق است یا خیر. می‌دانیم پاسخ الف به این سوال باعث می‌شود غریبه بتواند راستگو بودن یا دروغگو بودن الف و ب را به درستی تعیین کند، یعنی نتیجه بگیرد
  $\vDash p$
  و
  $\vDash q$
  برقرار هستند یا نه.
  
  همچنین می‌دانیم الف یکی از دو پاسخ زیر را به غریبه داده است.:
  \[ \text{پاسخ الف} =
    \begin{cases}
      \vDash p \vee q & (\text{i}) \\
      \not \vDash p \vee q & (\text{ii})
    \end{cases}
  \]

همچنین طبق فرض مسئله، الف می‌تواند راستگو یا دروغگو باشد. مسئله را با هر دو فرض بررسی می‌کنیم، و در هر یک از این دو حالت، دو پاسخی که الف ممکن است داده باشد را در نظر می‌گیریم.
  \begin{enumerate}
    \item فرض کنید الف دروغگو باشد.
    \begin{enumerate}
      \item فرض کنید پاسخ الف $\vDash p \vee q$
      باشد. چون الف دروغگو است داریم
      $\not \vDash p \vee q$.
      درنتیجه
      $\vDash \neg (p \vee q)$
      برقرار است. پس داریم
      $\vDash \neg p \wedge \neg q$.
      از طرفی می‌دانیم هر دو طرف یک ترکیب عطفیِ صادق باید صادق باشند. در نتیجه
      $\vDash \neg p$
      و
      $\vDash \neg q$.
      اما
      $\vDash \neg p$
      به معنی دروغگو نبودن الف است، که با فرض در تناقض است. بنابراین این حالت اتفاق نیافتاده است.
      
      \item فرض کنید پاسخ الف $\not \vDash p \vee q$
      باشد. چون الف دروغگو است، پس داریم
      $\vDash p \vee q$.
      می‌دانیم که لااقل یکی از طرفین یک ترکیب فصلیِ صادق باید صادق باشد، اما نمی‌توان در حالت کلی نتیجه گرفت کدام‌یک از طرفین صادق است. بنابراین در این حالت غریبه نمی‌تواند نتیجه‌ای درباره‌ی راستگویی الف و ب بگیرد و در نتیجه، این حالت اتفاق نیافتاده است.
    \end{enumerate}
    \item فرض کنید الف راستگو باشد.
    \begin{enumerate}
      \item فرض کنید پاسخ الف $\vDash p \vee q$
      باشد. مانند حالت قبل، چون نمی‌توان در حالت کلی نتیجه گرفت کدام‌یک از طرفین ترکیب فصلی صادق است، در این حالت نیز غریبه نمی‌تواند نتیجه‌ای درباره‌ی راستگویی الف و ب بگیرد. پس این حالت هم اتفاق نیافتاده است.

      \item فرض کنید پاسخ الف $\not \vDash p \vee q$
      باشد. درنتیجه
      $\vDash \neg (p \vee q)$
      برقرار است و داریم
      $\vDash \neg p \wedge \neg q$.
      همچنین می‌دانیم هر دو طرف یک ترکیب عطفی صادق باید صادق باشند. پس هر دوی
      $\vDash \neg p$
      و
      $\vDash \neg q$
      برقرار هستند، که به این معنی است که هر دوی الف و ب راستگو هستند.
    \end{enumerate}
  \end{enumerate}
  همان‌طور که مشاهده شد، تنها حالت سازگار، حالتی است که الف راستگو بوده و پاسخ او به پرسش غریبه «نه» باشد.
\end{ans}

\item
رابط سه‌موضعی $C(-,-,-)$ را به این صورت در نظر می‌گیریم: $C(X,Y,Z)$ یعنی اگر $X$ آنگاه $Y$، وگرنه $Z$. پس برای هر ارزیاب $v$ داریم:

$$
v(C(X,Y,Z))=
\begin{cases}
v(Y) & \text{if}~~v(X)=T\\
v(Z) & \text{if}~~v(X)=F
\end{cases}
$$
آیا $\{C\}$ کامل است؟ $\{\neg, C\}$ چطور؟
\begin{ans}
  ثابت می‌کنیم تابع بولی یک‌موضعی‌ای که به هر ورودی مقدار $F$ را نسبت می‌دهد قابل‌بیان با $\{C\}$ نیست و بنابراین این مجموعه کامل نیست. برای اثبات این حکم کافی است $v$ را ارزیابی در نظر بگیریم که به همه‌ی اتم‌ها مقدار $T$ را نسبت می‌دهد و با استدلالی مشابه استدلالی که برای پاسخ به پرسش ۵ (آ) استفاده کردیم نشان بدهیم این ارزیاب به تمام فرمول‌ها مقدار $T$ را نسبت می‌دهد.

  از آنجا که ثابت کرده‌ایم
  $\{\wedge,\neg\}$
  کامل است، کافی است نشان بدهیم هر $\{\wedge,\neg\}$-فرمول، هم‌ارز یک $\{C,\neg\}$-فرمول است. به‌سادگی می‌توان تحقیق کرد
  $C(A,B,A)$
  هم‌ارز با
  $(A\wedge B)$
  است. بنابراین کافی است با استقرا $\{\wedge,\neg\}$-فرمول‌ها را ترجمه کنیم.
\end{ans}

\item
فرض کنید $\Sigma$ و $\Gamma$ دو مجموعه از فرمول‌ها باشند. گوییم $\Gamma$ و $\Sigma$ هم‌ارزند اگر و تنها اگر مجموعه‌ی فرمول‌های یکسانی را ارضا کنند. همچنین گوییم $\Gamma$ مستقل است اگر برای هر $A\in\Gamma$ داشته باشیم
$$
\Gamma\textbackslash\{A\}\not\vDash A
$$
گزاره‌های زیر را اثبات یا رد کنید:
\begin{enumerate}
\item
هر مجموعه‌ی متناهی از فرمول‌ها دارای یک زیرمجموعه‌ی مستقل هم‌ارز با خودش است.
\item
هر مجموعه‌ی نامتناهی از فرمول‌ها دارای یک زیرمجموعه‌ی مستقل هم‌ارز با خودش است.
\item[(پ)]
برای هر مجموعه از فرمول‌ها مثل $\Gamma$ یک مجموعه‌ی مستقل از فرمول‌ها مثل $\Sigma$ وجود دارد که با $\Gamma$ هم‌ارز است.
\end{enumerate}\quad\vspace{-9mm}
\begin{ans}
  ابتدا توجّه کنید که می‌توان استقلال را به این صورت هم تعریف کرد: می‌گوییم $\Gamma$ مستقل است اگر $\Gamma$ زیرمجموعه‌ی سره‌ی هم‌ارز با خود نداشته باشد. معادل بودن این تعریف با تعریفی که در صورت سؤال آمده واضح است.
  \begin{enumerate}
    \item
    اگر مجموعه مستقل باشد حکم واضح است. پس حکم را برای مجموعه‌ی متناهی غیر مستقل اثبات می‌کنیم.

    از استقرای قوی روی اندازه‌ی مجموعه استفاده می‌کنیم. فرض کنید $\Gamma$ یک مجموعه‌ی غیر مستقل و حکم برای هر مجموعه‌ی کوچک‌تر از آن برقرار باشد.
    چون $\Gamma$ مستقل نیست، طبق تعریفِ معادل، باید زیرمجموعه‌ی سره‌ی هم‌ارزی مانند $\Delta$ داشته باشد. از آنجا که $\Delta$ کوچک‌تر از $\Gamma$ است، طبق فرض استقرا، زیرمجموعه‌ی مستقل هم‌ارزی مانند $\Delta'$ دارد. واضح است که $\Delta'$ زیرمجموعه‌ی $\Gamma$ و هم‌ارز با آن نیز هست.

    \item
    با مثال نقض رد می‌کنیم.

    مجموعه‌ی زیر را در نظر بگیرید.
    $$ \Gamma = \{p_1,~ p_1 \wedge p_2,~ p_1 \wedge p_2 \wedge p_3, \dots \} $$
    برای هر دو عضو از $\Gamma$، یکی دیگری را نتیجه می‌دهد. هیچ عضوی هم به تنهایی تمام $\Gamma$ را نتیجه نمی‌دهد. بنابراین $\Gamma$ زیرمجموعه‌ی مستقل هم‌ارز ندارد.

    \item
    فرض کنید
    $$ \Gamma = \{ A_n \mid 0 \leq n \} $$
    یک مجموعه‌ی نامتناهی باشد.
    گزاره‌ی $B_n$ و مجموعه‌ی $\Sigma$ را به صورت زیر را درنظر بگیرید.
    $$ B_n = (\bigwedge_{i=0}^{n-1} A_i) \rightarrow A_n $$
    $$ \Sigma = \{ B_n \mid 0 \leq n \} $$

    نشان می‌دهیم $\Gamma$ و $\Sigma$ هم‌ارز هستند.

    اعضای $\Sigma$ شرطی هستند و از تعریف ارزشدهی می‌دانیم برای درست بودن شرطی در یک ارزشدهی کافی است تالی آن شرطی در آن ارزشدهی درست باشد. حال ارزشدهی دلخواهی را فرض کنید که همه‌ی اعضای $\Gamma$ در آن درست هستند. واضح است که تالیِ همه‌ی اعضای $\Sigma$، و در نتیجه همه‌ی اعضای $\Sigma$ در آن تابع ارزش درست هستند. در نتیجه $\Gamma \vDash \Sigma$.

    برای اثبات $\Sigma \vDash \Gamma$، ابتدا
    مجموعه‌های $\Sigma_n$ و $\Gamma_n$ را به صورت زیر تعریف می‌کنیم.
    $$ \Sigma_n = \{ B_k \mid 0 \leq k \leq n \} $$
    $$ \Gamma_n = \{ A_k \mid 0 \leq k \leq n \} $$

    با استقرا روی $n$ نشان می‌دهیم برای هر $n$، $\Sigma_n \vDash \Gamma_n$.
    اگر $n=0$، واضح است که $\{A_0\} \vDash \{A_0\}$. فرض کنید $n=m+1$ و داشته باشیم $\Sigma_m \vDash \Gamma_m$. فرض کنید $v$ یک ارزشدهی دلخواه باشد که همه‌ی اعضای $\Sigma_{m+1} = \Sigma_m \cup \{ B_{m+1} \}$ در آن درست باشند. از فرض استقرا می‌دانیم همه‌ی اعضای $\Gamma_n$، یعنی $A_0$،...،$A_m$ هم در $v$ درست هستند. به عبارت دیگر مقدم $B_{m+1}$، و در نتیجه تالی آن، یعنی $A_{m+1}$ نیز در $v$ درست است. پس $\Gamma_{m+1}$ در $v$ درست است. بنابراین برای هر $n$ داریم $\Sigma_n \vDash \Gamma_n$. برای نشان دادن $\Sigma \vDash \Gamma$، تابع ارزش $v$ را طوری در نظر بگیرید که همه‌ی اعضای $\Sigma$ در آن درست باشند. فرض کنید $A_n$ عضو دلخواهی از $\Gamma$ باشد. از آن‌جا که همه‌ی اعضای $\Sigma$، و به طور خاص اعضای $\Sigma_n$ در $v$ درست هستند، طبق نتیجه‌ی قبل اعضای $\Gamma_n$، از جمله $A_n$ هم در $v$ درست هستند.

    اکنون $\tilde{\Sigma}$ را طوری تعریف کنید که شامل همه‌ی اعضای $\Sigma$ به جز توتولوژی‌های آن باشد. بدیهی است که $\tilde{\Sigma}$ با $\Sigma$، و در نتیجه با $\Gamma$ هم‌ارز است. حال کافی است نشان دهیم $\tilde{\Sigma}$ مستقل است.

    فرمول دلخواهی از $\tilde{\Sigma}$ مانند $B_n$ را در نظر بگیرید. از آنجا که $B_n$ توتولوژی نیست، پس باید یک ارزشدهی مانند $v$ موجود باشد که $v(B_n)$ نادرست است. در نتیجه، بنا به تعریف ارزشدهی روی شرطی، $v(A_n)$ نادرست و $v(\bigwedge_{i=0}^{n-1} A_i)$ درست است. از نادرست بودن $v(A_n)$ داریم برای همه‌ی $i$های بزرگ‌تر از $n$، مقدم $B_i$ در $v$ نادرست، و در نتیجه $B_i$ درست است. همچنین از درست بودن $v(\bigwedge_{i=0}^{n-1} A_i)$ داریم برای همه‌ی $i$های کوچک‌تر از $n$، تالی $B_i$ در $v$ درست، و درنتیجه $v(B_i)$ درست است. بنابراین برای هر $i$ که $i \neq n$، $v(B_i)$ درست است. پس $\tilde{\Sigma}\textbackslash\{B_n\} \not\vDash B_n$.

  \end{enumerate}
\end{ans}

\item
\begin{enumerate}
\item
ثابت کنید
$\neg(p_1\vee p_2)\Dashv\vDash\neg p_1\wedge\neg p_2$.

\item
با استفاده از قضایای ثابت‌شده در مبحث جایگزینی ثابت کنید
$$\neg((p_1\vee p_2)\vee p_3)\Dashv\vDash (\neg p_1 \wedge \neg p_2)\wedge\neg p_3$$

\item[(پ)]
با استقرا ثابت کنید به ازای هر $n$ داریم
$$\neg(\ldots(p_1\vee p_2)\vee \ldots)\vee p_n)\Dashv\vDash (\ldots(\neg p_1\wedge \neg p_2)\wedge\ldots)\wedge\neg p_n$$
\end{enumerate}\quad\vspace{-9mm}
\begin{ans}
  \begin{enumerate}
    \item جدول درسیتی گزاره‌های دو سمت هم‌ارزی را مجاسبه می‌کنیم.
    \begin{LTR}
      \begin{tabular}{| c | c | c | c |}
        $p_1$ & $p_2$ & $\neg (p_1 \vee p_2)$ & $\neg p_1 \wedge \neg p_2$ \\
        $F$ & $F$ & $T$ & $T$ \\
        $F$ & $T$ & $F$ & $F$ \\
        $T$ & $F$ & $F$ & $F$ \\
        $T$ & $T$ & $F$ & $F$
      \end{tabular}
    \end{LTR}
    دو ستون آخر یکسان هستند، در نتیجه هم‌ارزی برقرار است.

    \item
    می‌دانیم جایگزین کردن گزاره‌های هم‌ارز به جای یک اتم در یک گزاره، تغییری در ارزش آن گزاره نمی‌دهد. به عبارت دیگر اگر
    $A \Dashv\vDash B$ و
    $C \Dashv\vDash D$
    آنگاه
    $A[C/p_i] \Dashv\vDash B[D/p_i]$
    برای هر اتم $p_i$.

    هم‌ارزی اثبات‌شده در بخش قبل را در نظر بگیرید. ابتدا با جایگزینی‌های متوالی $p_2$ را به $p_3$ تغییر می‌دهیم.
    حال با جایگزینی
    $(p_1 \vee p_2)$
    به جای
    $p_1$
    خواهیم داشت
    $$ \neg ((p_1 \vee p_2) \vee p3) \Dashv\vDash (\neg (p_1 \vee p_2)) \wedge \neg p_3 $$
    حال گزاره‌ی
    $p_1 \wedge \neg p_3$
    را در نظر بگیرید. با جایگزینی دو سمت هم‌ارزی اثبات‌شده در بخش قبل در این گزاره به جای
    $p_1$
    داریم
    $$ (\neg (p_1 \vee p_2)) \wedge \neg p_3 \Dashv\vDash (\neg p_1 \wedge \neg p_2) \wedge \neg p_3 $$
    چون هم‌ارزی تراگذری است، می‌توان نتیجه گرفت
    $$ \neg((p_1\vee p_2)\vee p_3)\Dashv\vDash (\neg p_1 \wedge \neg p_2)\wedge\neg p_3 $$

    \item
    استقرا روی $n$:

    پایه‌ی استقرا:
    فرض کنیم $n = 3$. طبق بخش قبل حکم برقرار است.

    گام استقرا:
    فرض کنیم $n = m+1$ و حکم برای $m$ برقرار باشد.
    $$\neg(\ldots(p_1\vee p_2)\vee \ldots)\vee p_m)\Dashv\vDash (\ldots(\neg p_1\wedge \neg p_2)\wedge\ldots)\wedge\neg p_m$$
    ابتدا با جایگزینی‌های متوالی، همه‌ی اتم‌های $p_2$ تا $p_m$ را به ترتیب با $p_3$ تا $p_{m+1}$ جایگزین می‌کنیم.
    $$\neg(\ldots(p_1\vee p_3)\vee \ldots)\vee p_{m+1})\Dashv\vDash (\ldots(\neg p_1\wedge \neg p_3)\wedge\ldots)\wedge\neg p_{m+1}$$
    ادامه‌ی کار مانند بخش قبل خواهد بود. هم‌ارزی بخش اوّل را به جای $p_1$ در هم‌ارزی بالا جایگزین می‌کنیم.
    $$\neg(\ldots((p_1 \vee p_2) \vee p_3)\vee \ldots)\vee p_{m+1})\Dashv\vDash (\ldots((\neg (p_1 \vee p_2))\wedge \neg p_3)\wedge\ldots)\wedge\neg p_{m+1}$$
    حال گزاره‌ی زیر را در نظر بگیرید.
    $$(\ldots(p_1\wedge \neg p_3)\wedge\ldots)\wedge\neg p_{m+1}$$
    اگر دو سمت هم‌ارزی بخش اوّل را در این گزاره با $p_1$ جایگزین کنیم، خواهیم داشت
    $$ (\ldots((\neg (p_1 \vee p_2))\wedge \neg p_3)\wedge\ldots)\wedge\neg p_{m+1} \Dashv\vDash (\ldots((\neg p_1 \wedge \neg p_2) \wedge \neg p_3)\wedge\ldots)\wedge\neg p_{m+1}  $$
    از تراگذری بودن هم‌ارزی می‌توان نتیجه گرفت
    $$ \neg(\ldots((p_1 \vee p_2) \vee p_3)\vee \ldots)\vee p_{m+1}) \Dashv\vDash (\ldots((\neg p_1 \wedge \neg p_2) \wedge \neg p_3)\wedge\ldots)\wedge\neg p_{m+1}  $$
    پس حکم برای $n = m+1$ برقرار است.
  \end{enumerate}
\end{ans}

\end{enumerate}

\end{document}
