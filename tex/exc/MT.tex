\documentclass[12pt, 14paper]{article}

\usepackage{amsthm}
\usepackage{amsmath}
\usepackage{mathtools,array,booktabs,mathabx}
\usepackage{enumitem}
\usepackage{xepersian}

\settextfont[Scale=1.3]{IRBadr}
\linespread{1.2}


\title{آزمون میان‌ترم درس مبانی منطق}
\author{}
\date{}

\begin{document}

\maketitle

\vspace{-2.5cm}

\begin{center}آخرین زمان تحویل: ۱۶ فروردین‌ماه، ساعت ۲۳:۵۹\end{center}

\vspace{0.5cm}

در سؤالات یک تا پنج فرض کنید همه‌ی فرمول‌ها به زبان منطق گزاره‌ها هستند.

\begin{enumerate}

\item
فرض کنید رابط سه‌موضعی $A(X,Y,Z)$ را چنان تعریف کرده‌ایم که $A(X,Y,Z)$ صادق است اگر و فقط اگر دقیقاً یکی از $X$، $Y$ و $Z$ صادق باشد.

\begin{enumerate}
\item
با استفاده از رابط‌های
$\{\neg,\vee\}$
جمله‌ای بسازید معادل با
$A(X,Y,Z)$.
\item
نشان بدهید رابط دوموضعی $\square$ و $\circ$ای وجود ندارند چنانکه
$$
A(X,Y,Z)\Dashv\vDash(X\square Y)\circ Z
$$
\end{enumerate}

\begin{ans}
\begin{enumerate}
\item
$$
\neg(\neg X\vee Y\vee Z)\vee\neg(X\vee\neg Y\vee Z)\vee\neg(X\vee Y\vee\neg Z)
$$
\item
فرض کنید جدول درستی ادات
$\square$
چنین باشد:

\begin{tabular}{c|c|c}
$X$ & $Y$ & $X\square Y$ \\
$T$ & $T$ & $a$ \\
$T$ & $F$ & $b$ \\
$F$ & $T$ & $c$ \\
$F$ & $F$ & $d$ \\
\end{tabular}

حال باید داشته باشیم
$$
a\circ T = b\circ T = c\circ T = F \quad d\circ T = T
$$
با توجه به اینکه تنها دو ارزش صدق داریم، می‌توانیم نتیجه بگیریم
$a=b=c$
و
$a\neq d$.
اما واضح است که
$a\circ F=F$
در حالی که
$b\circ F=T$
که نتیجه می‌دهد
$a\neq b$.
در نتیجه چنین دو رابطی وجود ندارند.

\end{enumerate}
\end{ans}

\item
برای $n$ دلخواه ثابت کنید
$$
A\vee (B_1\wedge\ldots\wedge B_n)\Dashv\vDash(A\vee B_1)\wedge\ldots\wedge(A\vee B_n)
$$

\begin{ans}
فرض می‌کنیم همه‌ی فرمول‌های $A$ و $B_1$ تا $B_n$ اتم‌اند. واضح است که اگر حکم را در این حالت ثابت کنیم، حکم در باقی حالت‌ها نیز ثابت شده است چرا که هر مصداق از یک توتولوژی توتولوژی است.

برای اثبات حکم از استقرا استفاده می‌کنیم. برای پایه‌ی استقرا قرار دهید $n=2$. در این حالت کافی است جدول درستی دو سوی هم‌ارزی را رسم کنیم:

\begin{tabular}{c|c|c|c|c}
$A$ & $B_1$ & $B_2$ & $A\vee(B_1\wedge B_2)$ & $(A\vee B_1)\wedge(A\vee B_2)$ \\
$T$ & $T$ & $T$ & $T$ & $T$ \\
$T$ & $T$ & $F$ & $T$ & $T$ \\
$T$ & $F$ & $T$ & $T$ & $T$ \\
$T$ & $F$ & $F$ & $T$ & $T$ \\
$F$ & $T$ & $T$ & $T$ & $T$ \\
$F$ & $T$ & $F$ & $F$ & $F$ \\
$F$ & $F$ & $T$ & $F$ & $F$ \\
$F$ & $F$ & $F$ & $F$ & $F$ \\
\end{tabular}

حال فرض کنید حکم برای $n=k$ ثابت شده است. این هم‌ارزی از حالت پایه‌ی استقرا نتیجه می‌شود:
$$
A\vee ((B_1\wedge\ldots\wedge B_k)\wedge B_{k+1})\Dashv\vDash (A\vee (B_1\wedge\ldots\wedge B_k)\wedge (A\vee B_{k+1})
$$
نیز، با استفاده از قضایای ثابت‌شده در مبحث جایگزینی و مفروض بودن حکم برای $n=k$ می‌توانیم نتیجه بگیریم:
$$
(A\vee (B_1\wedge\ldots\wedge B_k)\wedge (A\vee B_{k+1})\Dashv\vDash(A\vee B_1)\wedge\ldots(A\vee B_k)\wedge(A\vee B_{k+1})
$$

\end{ans}

\item
\begin{enumerate}
\item
فرض کنید $A$ فصلی از لیترال‌ها باشد. الگوریتمی ارائه کنید که ابتدا بررسی کند آیا $A$ یک توتولوژی هست یا نه و در صورتی که $A$ یک توتولوژی باشد استنتاجی برای آن در دستگاه استنتاج طبیعی گنتزن ارائه کند.
\item
فرض کنید $A$ فرمولی در صورت نرمال عطفی (\lr{CNF}) باشد. الگوریتمی ارائه کنید که ابتدا بررسی کند آیا $A$ یک توتولوژی هست یا نه و در صورتی که $A$ یک توتولوژی باشد استنتاجی برای آن در دستگاه استنتاج طبیعی گنتزن ارائه کند.
\end{enumerate}

\item
الگوریتمی طراحی کنید که اگر برای مدتی نامتناهی به اجرا گذاشته شود همه‌ی توتولوژی‌های منطق گزاره‌ها را خروجی دهد. ثابت کنید الگوریتمتان کار خواسته‌شده را به‌درستی انجام می‌دهد.

\begin{ans}
تابع $g$ را به شکل زیر تعریف می‌کنیم:
$$
g(A)=
\begin{cases}
2\cdot 3^{n+1} & \text{if}~~A = p_n\\
2^2\cdot 3^{g(A_1)} & \text{if}~~A=(\neg A_1)\\
2^3\cdot 3^{g(A_1)}\cdot 5^{g(A_2)} & \text{if}~~A=(A_1\wedge A_2) \\
2^4\cdot 3^{g(A_1)}\cdot 5^{g(A_2)} & \text{if}~~A=(A_1\vee A_2) \\
2^5\cdot 3^{g(A_1)}\cdot 5^{g(A_2)} & \text{if}~~A=(A_1\rightarrow A_2)
\end{cases}
$$
واضح است که این تابع به هر فرمول عددی یکتا نسبت می‌دهد. اگر یک عدد ورودی بگیریم، به‌سادگی می‌توانیم با تجزیه‌ی آن عدد به عوامل اول بررسی کنیم آیا این عدد در یکی از عبارات تعریف $g$ قرار می‌گیرد یا نه، و در صورت لزوم همین کار را درباره‌ی توان عوامل اول آن نیز انجام بدهیم تا نهایتاً تعیین کنیم آیا فرمولی وجود دارد که تابع $g$ به آن این عدد را نسبت بدهد یا نه و اگر چنین فرمولی وجود دارد این فرمول چیست. پس از بازسازی فرمول، به‌سادگی می‌توان اتم‌های آن را مشخص کرد و با ترسیم جدول‌ارزش توتولوژی بودن آن را تعیین کرد. واضح است که برای هر عدد طبیعی هر دوی این مراحل در زمان متناهی پایان می‌پذیرند. حال یک رایانه می‌تواند به‌ترتیب این فرمول را روی همه‌ی اعداد طبیعی از کوچک به بزرگ انجام بدهد و هر گاه به عددی می‌رسد که فرمول متناظرش توتولوژی است آن را چاپ کند. این برنامه اگر به مدت نامتناهی به اجرا گذاشته شود همه‌ی توتولوژی‌ها را خروجی می‌دهد زیرا به ازای هر توتولوژی $A$، این الگوریتم نهایتاً به
$g(A)$
می‌رسد و پس از بازسازی $A$ از روی $g(A)$ و بررسی توتولوژی بودنش آن را خروجی می‌دهد.
\end{ans}

\item
در سال ۱۹۷۶ کنت آپل و ولفگانگ هاکن ثابت کردند هر نقشه‌ی متناهی را می‌توان با چهار رنگ رنگ‌آمیزی کرد، یعنی با چهار رنگ می‌توان هر نقشه‌ی متناهی را چنان رنگ‌آمیزی کرد که هیچ دو کشور همجوار هم‌رنگ نباشند. (یکی از دلایل شهرت این قضیه، قضیه‌ی چهار رنگ، استفاده‌ی گسترده‌ی ریاضی‌دانان از رایانه در اثبات آن است.) با فرض گرفتن این قضیه برای نقشه‌های متناهی ثابت کنید همین حکم درباره‌ی نقشه‌های نامتناهی نیز صادق است.


\item
با استفاده از استدلال‌های معناشناختی نشان بدهید فرمول‌های زیر معتبر هستند یا نه. برای فرمول‌هایی که معتبر نیستند مدل نقض ارائه کنید.
\begin{enumerate}
\item
$(\exists x\forall y R(x,y))\leftrightarrow(\forall y \exists x R(x,y))$
\item
$\exists x(P(x)\to \forall y P(y))$
\end{enumerate}

\item
فرض کنید زبان
$\mathcal{L}$
تنها دارای نماد تساوی است. مجموعه‌ی $\Gamma$ از جملات زبان $\mathcal{L}$ را ارائه کنید که به ازای هر عدد طبیعی $k$ مدلی با $2k$ عضو داشته باشد اما مدلی با $2k+1$ عضو نداشته باشد.

\end{enumerate}

\end{document}
