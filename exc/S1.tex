\documentclass[12pt, 14paper]{article}

\usepackage{amsthm}
\usepackage{amsmath, amsfonts}
\usepackage{mathtools,array,booktabs,mathabx}
\usepackage{natbib}
\usepackage{forest}

\usepackage{xepersian}

\settextfont[Scale=1.3]{IRBadr}
\linespread{1.2}

\usepackage{versions}
\excludeversion{ans}



\renewcommand\beginmarkversion{\\\textbf{پاسخ: }}
\renewcommand\endmarkversion{$\qed$}

\markversion{ans}

\title{سری اول تمرینات درس مبانی منطق}
\author{}
\date{}

\begin{document}

\maketitle

\vspace{-2.5cm}

\begin{center}آخرین زمان تحویل: ۲۴ اسفندماه، ساعت ۲۳:۵۹\end{center}

\vspace{0.5cm}

\begin{enumerate}

\item
ثابت کنید مجموعه‌ی تمام عبارات زبان منطق گزاره‌ها با مجموعه‌ی اتم‌های شمارا، شمارا است.
\begin{ans}
  مطابق قضیه‌ی کانتور-شرودر-برنشتاین کافی است تابعی یک‌به‌یک از $\mathbb{N}$ به توی مجموعه‌ی فرمول‌ها و تابعی یک‌به‌یک از مجموعه‌ی فرمول‌ها به توی $\mathbb{N}$ معرفی کنیم. برای تابع اول این تابع را در نظر بگیرید:
  $$
  f(n)=p_n
  $$
  یک‌به‌یک بودن  این تابع واضح است. برای تابع دوم این تابع را در نظر بگیرید (چنین تابعی را یک \emph{عددگذاری گودلی} می‌نامیم):
  $$
  g(A)=
  \begin{cases}
  2\cdot 3^{n+1} & \text{if}~~A = p_n\\
  2^2\cdot 3^{g(A_1)} & \text{if}~~A=(\neg A_1)\\
  2^3\cdot 3^{g(A_1)}\cdot 5^{g(A_2)} & \text{if}~~A=(A_1\wedge A_2) \\
  2^4\cdot 3^{g(A_1)}\cdot 5^{g(A_2)} & \text{if}~~A=(A_1\vee A_2) \\
  2^5\cdot 3^{g(A_1)}\cdot 5^{g(A_2)} & \text{if}~~A=(A_1\rightarrow A_2)
  \end{cases}
  $$
  با استفاده از قضیه‌ی اساسی حساب می‌توان نشان داد تنها در صورتی $g(A)=g(B)$ که $A=B$ و بنابراین $g$ یک‌به‌یک است.  
\end{ans}
\item
برای هر یک از رشته‌های زیر یا درخت تجزیه‌ی آن را رسم کنید یا از طریق تلاش برای رسم درخت تجزیه نشان بدهید آن رشته یک فرمول نیست:
\begin{enumerate}
\item $((((\neg p_1)\rightarrow\bot)\vee p_1)\wedge p_2)$
\item $(((\neg (p_0\vee p_1))\wedge(p_2\rightarrow p_3)))\rightarrow (p_3\wedge p_4))$
\item[(پ)] $(p_1\wedge\rightarrow\neg(p_2\vee p_0))$
\end{enumerate}

\item
ثابت کنید:
\begin{enumerate}
\item
اگر $c$ تعداد جایگاه‌هایی در فرمول $A$ باشد که رابطی دوتایی در آن قرار گرفته و $s$ تعداد جایگاه‌هایی در $A$ باشد که یک اتم در آن قرار گرفته، داریم $s=c+1$.
\item
اگر $A$ فرمولی باشد که در آن از $\neg$ استفاده نشده است، طول $A$ فرد است.
\end{enumerate}
\quad
\begin{ans}
  \begin{enumerate}
    \item
    حکم را از طریق استقرا ثابت می‌کنیم. اگر $A$ اتم باشد حکم بدیهی است. حال فرض کنید حکم برای فرمول‌هایی با پیچیدگی کمتر از پیچیدگی $A$ ثابت شده است. اگر $A=(\neg A_1)$ و $c_1$ تعداد جایگاه‌های رابط‌های دوتایی در $A_1$ و $s_1$ تعداد جایگاه‌های اتم‌ها در $A_1$ باشد، واضح است که $s=s_1$ و $c=c_1$ و بنابراین $s=c+1$. همچنین اگر $A=(A_1\square A_2)$ (که در آن $\square$ رابطی دوتایی است) و $c_i$ تعداد جایگاه‌های رابط‌های دوتایی در $A_i$ و $s_i$ تعداد جایگاه‌های اتم‌ها در $A_i$ باشد، داریم $c=c_1+c_2+1$ و $s=s_1+s_2$. حال با توجه به اینکه فرض کرده‌ایم $s_i=c_i+1$، به‌سادگی می‌توان نشان داد $s=c+1$.
    
    
    \item
    حکم را از طریق استقرا ثابت می‌کنیم. اگر $A$ اتم باشد حکم بدیهی است. حال فرض کنید حکم برای فرمول‌هایی با پیچیدگی کمتر از پیچیدگی $A$ ثابت شده است. تنها لازم است حالتی را بررسی کنیم که در آن $A=(A_1\square A_2)$ (که در آن $\square$ رابطی دوتایی است) زیرا در صورتی که $A=(\neg A_1)$ فرض استفاده نشدن از نقیض برقرار نیست. حال واضح است که اگر طول $A_1$ و $A_1$ فرد باشد طول $A$ نیز فرد است.
    
  \end{enumerate}
\end{ans}

\item
فرض کنید $\models A\rightarrow B$ و $A$ و $B$ دارای اتم‌های مشترک نیستند. ثابت کنید یا $A$ ارضاناپذیر است یا $B$ توتولوژی است (یا هر دو). توضیح بدهید که آیا شرط اتم مشترک نداشتن برای این حکم ضروری است یا نه.
\begin{ans}
  با استفاده از قضیه‌ی استنتاج می‌توانیم نتیجه بگیریم $A\models B$. بنابراین هر ارزیاب $v$ که $A$ را ارضا کند، $B$ را نیز ارضا می‌کند. برای اثبات حکم کافی است فرض کنیم $A$ ارضاناپذیر نیست و نتیجه بگیریم $B$ توتولوژی است. اگر $A$ ارضاناپذیر نباشد، ارزیاب $v$ای وجود دارد که $A$ را ارضا می‌کند. حال یک ارزیاب دلخواه $u$ را در نظر می‌گیریم و نشان می‌دهیم $B$ را ارضا می‌کند. ارزیاب $v'$ را به شکل زیر تعریف می‌کنیم:

  $$
  v'(p_n)=
  \begin{cases}
  u(p_n) & \text{if}~~p_n\in atoms(B) \\
  v(p_n) & \text{otherwise}
  \end{cases}
  $$

  حال واضح است که $v'$ نیز همانند $v$ فرمول $A$ را ارضا می‌کند زیرا مقدار این دو ارزیاب در اتم‌های موجود در $A$ یکسان است و در جزوه ثابت کرده‌ایم اگر دو ارزیاب مقادیر یکسانی به اتم‌های داخل یک فرمول نسبت بدهند، به فرمول نیز مقدار یکسانی نسبت می‌دهند. بنابراین از آنجا که $v'$ فرمول $A$ را ارضا می‌کند فرمول $B$ را نیز ارضا می‌کند. نیز، از آنجا که مقداری که ارزیاب $v'$ به اتم‌های $B$ نسبت می‌دهد همان مقداری است که ارزیاب $u$ به آن‌ها نسبت می‌دهد، مقداری که این دو ارزیاب به $B$ نسبت می‌دهند یکسان است. بنابراین ارزیاب $u$ نیز $B$ را ارضا می‌کند. با توجه به اینکه ارزیاب $u$ دلخواه است، هر ارزیابی $B$ را ارضا می‌کند و بنابراین $B$ توتولوژی است.

  در صورتی که شرط مشترک نبودن اتم‌ها را حذف کنیم حکم برقرار نیست؛ مثلاً
  $\models p\rightarrow p$
  اما $p$ نه ارضاناپذیر است و نه توتولوژی.
\end{ans}

\item
اثبات یا رد کنید:

\begin{enumerate}
\item
اگر $A$ یک $\{\wedge,\vee,\rightarrow,\leftrightarrow\}$-فرمول باشد، آنگاه $A$ ارضاشدنی است.
\item
اگر $A$ یک $\{\wedge,\vee\}$-فرمول باشد، آنگاه $A$ توتولوژی نیست.
\end{enumerate}
\quad
\begin{ans}
\begin{enumerate}
  \item
  فرض کنید $v$ ارزیابی باشد که به همه‌ی اتم‌ها مقدار $T$ را نسبت می‌دهد. نشان می‌دهیم $v(A)=T$. اگر $A$ اتم باشد حکم واضح است. حال فرض کنید $A$ اتم نیست و حکم درباره‌ی فرمول‌هایی با پیچیدگی کمتر ثابت شده است. در این صورت یا $A=(A_1\square A_2)$ ($\square\in\{\wedge,\vee,\rightarrow,\leftrightarrow\}$) و $v(A_1)=v(A_2)=T$. واضح است که برای هر چهار رابط $v(A)=T$. بنابراین $A$ تحت لااقل یک ارزیاب صادق است و ارضاناپذیر نیست.
  
  \item
  فرض کنید $v$ ارزیابی باشد که به همه‌ی اتم‌ها مقدار $F$ را نسبت می‌دهد. نشان می‌دهیم $v(A)=F$. اگر $A$ اتم باشد حکم واضح است. حال فرض کنید $A$ اتم نیست و حکم درباره‌ی فرمول‌هایی با پیچیدگی کمتر ثابت شده است. در این صورت یا $A=(A_1\wedge A_2)$ یا $A=(A_1\vee A_2)$ و $v(A_1)=v(A_2)=F$. واضح است که در هر دو حالت $v(A)=F$. بنابراین $A$ تحت لااقل یک ارزیاب کاذب است و توتولوژی نیست.
  
  \end{enumerate}
\end{ans}

\item
شخصی غریبه به جزیره‌ای وارد می‌شود که برخی ساکنان آن همیشه دروغ می‌گویند و برخی ساکنان آن همیشه راست. غریبه به دو ساکن جزیره با نام‌های الف و ب می‌رسد و از الف می‌پرسد که «آیا یکی از شما دو نفر دروغگو است؟» الف پاسخی بله/خیر می‌دهد که باعث می‌شود غریبه بتواند راستگو بودن یا دروغگو بودن هر دو نفر را تعیین کند. با نوشتن گزاره‌ی مورد سؤال («آیا یکی از شما دو نفر دروغگو است؟») به زبان صوری و بررسی شرایط صدق آن، مشخص کنید پاسخ الف چه بوده است.

\item
رابط سه‌موضعی $C(-,-,-)$ را به این صورت در نظر می‌گیریم: $C(X,Y,Z)$ یعنی اگر $X$ آنگاه $Y$، وگرنه $Z$. پس برای هر ارزیاب $v$ داریم:

$$
v(C(X,Y,Z))=
\begin{cases}
v(Y) & \text{if}~~v(X)=T\\
v(Z) & \text{if}~~v(X)=F
\end{cases}
$$
آیا $\{C\}$ کامل است؟ $\{\neg, C\}$ چطور؟
\begin{ans}
  ثابت می‌کنیم تابع بولی یک‌موضعی‌ای که به هر ورودی مقدار $F$ را نسبت می‌دهد قابل‌بیان با $\{C\}$ نیست و بنابراین این مجموعه کامل نیست. برای اثبات این حکم کافی است $v$ را ارزیابی در نظر بگیریم که به همه‌ی اتم‌ها مقدار $T$ را نسبت می‌دهد و با استدلالی مشابه استدلالی که برای پاسخ به پرسش ۵ (آ) استفاده کردیم نشان بدهیم این ارزیاب به تمام فرمول‌ها مقدار $T$ را نسبت می‌دهد.

  از آنجا که ثابت کرده‌ایم
  $\{\wedge,\neg\}$
  کامل است، کافی است نشان بدهیم هر $\{\wedge,\neg\}$-فرمول، هم‌ارز یک $\{C,\neg\}$-فرمول است. به‌سادگی می‌توان تحقیق کرد
  $C(A,B,A)$
  هم‌ارز با
  $(A\wedge B)$
  است. بنابراین کافی است با استقرا $\{\wedge,\neg\}$-فرمول‌ها را ترجمه کنیم.
\end{ans}

\item
فرض کنید $\Sigma$ و $\Gamma$ دو مجموعه از فرمول‌ها باشند. گوییم $\Gamma$ و $\Sigma$ هم‌ارزند اگر و تنها اگر مجموعه‌ی فرمول‌های یکسانی را ارضا کنند. همچنین گوییم $\Gamma$ مستقل است اگر برای هر $A\in\Gamma$ داشته باشیم
$$
\Gamma\textbackslash\{A\}\not\vDash A
$$
گزاره‌های زیر را اثبات یا رد کنید:
\begin{enumerate}
\item
هر مجموعه‌ی متناهی از فرمول‌ها دارای یک زیرمجموعه‌ی مستقل هم‌ارز با خودش است.
\item
هر مجموعه‌ی نامتناهی از فرمول‌ها دارای یک زیرمجموعه‌ی مستقل هم‌ارز با خودش است.
\item[(پ)]
برای هر مجموعه از فرمول‌ها مثل $\Gamma$ یک مجموعه‌ی مستقل از فرمول‌ها مثل $\Sigma$ وجود دارد که با $\Gamma$ هم‌ارز است.
\end{enumerate}

\item
\begin{enumerate}
\item
ثابت کنید
$\neg(p_1\vee p_2)\Dashv\vDash\neg p_1\wedge\neg p_2$.

\item
با استفاده از قضایای ثابت‌شده در مبحث جایگزینی ثابت کنید
$$\neg((p_1\vee p_2)\vee p_3)\Dashv\vDash (\neg p_1 \wedge \neg p_2)\wedge\neg p_3$$

\item[(پ)]
با استقرا ثابت کنید به ازای هر $n$ داریم
$$\neg(\ldots(p_1\vee p_2)\vee \ldots)\vee p_n)\Dashv\vDash (\ldots(\neg p_1\wedge \neg p_2)\wedge\ldots)\wedge\neg p_n$$

\end{enumerate}

\end{enumerate}

\end{document}
