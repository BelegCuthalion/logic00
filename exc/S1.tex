\documentclass[12pt, 14paper]{article}

\usepackage{amsthm}
\usepackage{amsmath}
\usepackage{mathtools,array,booktabs,mathabx}
\usepackage{natbib}
\usepackage{xepersian}

\settextfont[Scale=1.3]{IRBadr}
\linespread{1.2}


\title{سری اول تمرینات درس مبانی منطق}
\author{}
\date{}

\begin{document}

\maketitle

\section{تمرینات نظری}

\begin{enumerate}

\item
ثابت کنید مجموعه‌ی تمام عبارات زبان منطق گزاره‌ها با مجموعه‌ی اتم‌های شمارا، شمارا است.

\item
برای هر یک از رشته‌های زیر یا درخت تجزیه‌ی آن را رسم کنید یا از طریق تلاش برای رسم درخت تجزیه نشان بدهید آن رشته یک فرمول نیست:
\begin{enumerate}
\item $((((\neg p_1)\rightarrow\bot)\vee p_1)\wedge p_2)$
\item $(((\neg (p_0\vee p_1))\wedge(p_2\rightarrow p_3)))\rightarrow (p_3\wedge p_4))$
\item[(پ)] $(p_1\wedge\rightarrow\neg(p_2\vee p_0))$
\end{enumerate}

\item
ثابت کنید:
\begin{enumerate}
\item
اگر $c$ تعداد جایگاه‌هایی در فرمول $A$ باشد که رابطی دوتایی در آن قرار گرفته و $s$ تعداد جایگاه‌هایی در $A$ باشد که یک اتم در آن قرار گرفته، داریم $s=c+1$.
\item
اگر $A$ فرمولی باشد که در آن از $\neg$ استفاده نشده است، طول $A$ فرد است.
\end{enumerate}

\item
فرض کنید $\models A\rightarrow B$ و $A$ و $B$ دارای اتم‌های مشترک نیستند. ثابت کنید یا $A$ ارضاناپذیر است یا $B$ توتولوژی است (یا هر دو). توضیح بدهید که آیا شرط اتم مشترک نداشتن برای این حکم ضروری است یا نه.

\pagebreak

\item
اثبات یا رد کنید:

\begin{enumerate}
\item
اگر $A$ یک $\{\wedge,\vee,\rightarrow,\leftrightarrow\}$-فرمول باشد، آنگاه $A$ ارضاشدنی است.
\item
اگر $A$ یک $\{\wedge,\vee\}$-فرمول باشد، آنگاه $A$ توتولوژی نیست.
\end{enumerate}

\item
شخصی غریبه به جزیره‌ای وارد می‌شود که برخی ساکنان آن همیشه دروغ می‌گویند و برخی ساکنان آن همیشه راست. غریبه به دو ساکن جزیره با نام‌های الف و ب می‌رسد و از الف می‌پرسد که «آیا یکی از شما دو نفر دروغگو است؟» الف پاسخی بله/خیر می‌دهد که باعث می‌شود غریبه بتواند راستگو بودن یا دروغگو بودن هر دو نفر را تعیین کند. با نوشتن گزاره‌ی مورد سؤال («آیا یکی از شما دو نفر دروغگو است؟») به زبان صوری و بررسی شرایط صدق آن، مشخص کنید پاسخ الف چه بوده است.

\item
رابط سه‌موضعی $C(-,-,-)$ را به این صورت در نظر می‌گیریم: $C(X,Y,Z)$ یعنی اگر $X$ آنگاه $Y$، وگرنه $Z$. پس برای هر ارزیاب $v$ داریم:

$$
v(C(X,Y,Z))=
\begin{cases}
v(Y) & \text{if}~~v(X)=T\\
v(Z) & \text{if}~~v(X)=F
\end{cases}
$$
آیا $\{C\}$ کامل است؟ $\{\neg, C\}$ چطور؟

\item
فرض کنید $\Sigma$ و $\Gamma$ دو مجموعه از فرمول‌ها باشند. گوییم $\Gamma$ و $\Sigma$ هم‌ارزند اگر و تنها اگر مجموعه‌ی فرمول‌های یکسانی را ارضا کنند. همچنین گوییم $\Gamma$ مستقل است اگر برای هر $A\in\Gamma$ داشته باشیم
$$
\Gamma\textbackslash\{A\}\not\vDash A
$$
گزاره‌های زیر را اثبات یا رد کنید:
\begin{enumerate}
\item
هر مجموعه‌ی متناهی از فرمول‌ها دارای یک زیرمجموعه‌ی مستقل هم‌ارز با خودش است.
\item
هر مجموعه‌ی نامتناهی از فرمول‌ها دارای یک زیرمجموعه‌ی مستقل هم‌ارز با خودش است.
\item[(پ)]
برای هر مجموعه از فرمول‌ها مثل $\Gamma$ یک مجموعه‌ی مستقل از فرمول‌ها مثل $\Sigma$ وجود دارد که با $\Gamma$ هم‌ارز است.
\end{enumerate}

\item
\begin{enumerate}
\item
ثابت کنید
$\neg(p_1\vee p_2)\Dashv\vDash\neg p_1\wedge\neg p_2$.

\item
با استفاده از قضایای ثابت‌شده در مبحث جایگزینی ثابت کنید
$$\neg((p_1\vee p_2)\vee p_3)\Dashv\vDash (\neg p_1 \wedge \neg p_2)\wedge\neg p_3$$

\item[(پ)]
با استقرا ثابت کنید به ازای هر $n$ داریم
$$\neg(\ldots(p_1\vee p_2)\vee \ldots)\vee p_n)\Dashv\vDash (\ldots(\neg p_1\wedge \neg p_2)\wedge\ldots)\wedge\neg p_n$$

\end{enumerate}

\end{enumerate}

\section{تمرینات برنامه‌نویسی}

\begin{enumerate}
\item
برنامه‌ای بنویسید که یک رشته را ورودی بگیرد و تعیین کند آیا آن رشته فرمول هست یا نه.
\item
برنامه‌ای بنویسید که یک فرمول ورودی بگیرد و ارتفاع درخت تجزیه‌ی آن را تعیین کند و زیرفرمول‌های ورودی را نیز چاپ کند.
\end{enumerate}

\end{document}
