\documentclass[12pt, 14paper]{article}

\usepackage{amsthm}
\usepackage{amsmath, amsfonts}
\usepackage{mathtools,array,booktabs,mathabx}
\usepackage{natbib}
\usepackage{forest}

\usepackage{xepersian}

\settextfont[Scale=1.3]{IRBadr}
\linespread{1.2}

\usepackage{versions}
\excludeversion{ans}



\renewcommand\beginmarkversion{\\\textbf{پاسخ: }}
\renewcommand\endmarkversion{$\qed$}

\title{سری دوم تمرینات درس مبانی منطق}
\author{}
\date{}

\begin{document}

\maketitle

\vspace{-2.5cm}

\begin{center}آخرین زمان تحویل: ۲۴ اسفندماه، ساعت ۲۳:۵۹\end{center}

\vspace{0.5cm}

\begin{enumerate}

\item
استنتاج‌های زیر را با دستگاه اصل‌موضوعی هیلبرت ثابت کنید.

\begin{enumerate}
\item
$\vdash \neg\neg A\to A$

\item
$\vdash (A\to\neg A)\to \neg A$
\end{enumerate}

\item
اصول دستگاه اصل‌موضوعی هیلبرت را در دستگاه استنتاج طبیعی گنتزن ثابت کنید.

\item
توتولوژی
$(((p\to q)\to p)\to p)$
مشهور به قانون پِرس
(\lr{Peirce's law})
است.
\begin{enumerate}
\item
توتولوژی بودن قانون پرس را با استفاده از دستگاه استنتاج طبیعی گنتزن ثابت کنید.
\item
فرض کنید به جای دو ارزش‌صدق $T$ و $F$ مجموعه‌ی ارزش‌های صدق ما مجموعه‌ی $\{0,\frac{1}{2},1\}$ باشد. اگر ارزش‌صدق فرمول $A$ برابر $i$ و ارزش‌صدق فرمول $B$ برابر $j$ باشد، ارزش‌صدق $A\to B$ را بزرگترین عدد حقیقی $r\leq 1$ تعریف می‌کنیم که
$\text{min}\{r,i\}\leq j$.
جدول ارزش $\to$ را برای هر نه حالت ممکن از ارزش‌صدق طرفین آن رسم کنید.
\item
مقادیری از $p$ و $q$ را بیابید که برای آن‌ها ارزش‌صدق $(((p\to q)\to p)\to p)$ برابر $1$ نیست.
\item
نشان بدهید هر استنتاج $D$ که تنها از قانون
\lr{(Axiom)}
و قوانین معرفی و حذف $\to$ استفاده کند، دارای این ویژگی است: اگر $A$ نتیجه‌ی $D$ باشد و $v$ ارزیابی باشد که $v(A)<1$، مقدمه‌ی $B$ای در $D$ هست چنانکه $v(B)\leq v(A)$.
\item
با استفاده از نتایج به‌دست‌آمده در بخش‌های قبل ثابت کنید قانون پرس را نمی‌توان تنها با استفاده از قانون
\lr{(Axiom)}
و قوانین معرفی و حذف $\to$ ثابت کرد.
\end{enumerate}

\item
فرض کنید $\Sigma$ مجموعه‌ای از فرمول‌ها باشد چنانکه به ازای هر اتم $p$ یا
$p\in\Sigma$
یا
$\neg p\in\Sigma$.
اگر قرار بدهیم
$$
\Sigma' = \{A|\Sigma\vdash A\}
$$
نشان بدهید $\Sigma'$ کامل است.
\item
گوییم فرمول $A$ از مجموعه‌ی $\Sigma$ مستقل است اگر نه $\Sigma\vdash A$ و نه $\Sigma\vdash\neg A$. نشان دهید فرمول $p_0\to p_1$ از مجموعه‌ی
$\{p_0\to p_2, p_1\to p_2, p_2\}$
مستقل است.

\item
ثابت کنید هر مجموعه‌ی ارضاپذیر، سازگار است.

\item
مجموعه‌ی نامتناهی
$\{A_1,A_2,\ldots\}$
را در نظر بگیرید. فرض کنید برای هر ارزیاب $v$ عدد $n$ وجود دارد چنانکه $v(A_n)=T$. ثابت کنید عدد $m$ وجود دارد که
$\models A_1\vee\ldots A_m$.
(راهنمایی: از قضیه‌ی فشردگی استفاده کنید.)
\item
فرض کنید
$\Sigma_1$
و
$\Sigma_2$
دو مجموعه از فرمول‌ها باشند که
$\Sigma_1\cup \Sigma_2$
ناسازگار است. ثابت کنید فرمول $A$ موجود است که $\Sigma_1\vdash A$ و $\Sigma_2\vdash \neg A$.

\item
برای هر فرمول $A$ و اتم $p$ فرض کنید
$A^*=A[p/\top]\vee A[p/\bot]$.
ثابت کنید:
\begin{enumerate}
\item
$A\models A^*$
\item
اگر
$A\models B$
و $p$ در $B$ موجود نباشد، آنگاه
$A^*\models B$.
\item
(قضیه‌ی درونیابی کریگ) اگر
$A\models B$
آنگاه فرمولی مانند $C$ موجود است که اتم‌های آن مشترک میان $A$ و $B$ است و $A\models C$ و $C\models B$.
\end{enumerate}

\end{enumerate}

\end{document}
