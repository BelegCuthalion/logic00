\documentclass[12pt, 14paper]{article}

\usepackage{amsthm}
\usepackage{amsmath}
\usepackage{mathtools,array,booktabs,mathabx}
\usepackage{enumitem}
\usepackage{xepersian}

\settextfont[Scale=1.3]{IRBadr}
\linespread{1.2}


\title{سری دوم تمرینات درس مبانی منطق}
\author{}
\date{}

\begin{document}

\maketitle

\vspace{-2.5cm}

\begin{center}آخرین زمان تحویل: ۱۶ فروردین‌ماه، ساعت ۲۳:۵۹\end{center}

\vspace{0.5cm}

\begin{enumerate}

\item
استنتاج‌های زیر را در دستگاه اصل‌موضوعی هیلبرت ثابت کنید.

\begin{enumerate}
\item
$\vdash \neg\neg A\to A$

\item
$\vdash (A\to\neg A)\to \neg A$
\end{enumerate}

\item
توتولوژی
$(((p\to q)\to p)\to p)$
مشهور به قانون پِرس
(\lr{Peirce's law})
است.
\begin{enumerate}
\item
توتولوژی بودن قانون پرس را با استفاده از دستگاه استنتاج طبیعی گنتزن ثابت کنید.
\item
فرض کنید به جای دو ارزش‌صدق $T$ و $F$ مجموعه‌ی ارزش‌های صدق ما مجموعه‌ی $\{0,\frac{1}{2},1\}$ باشد. اگر ارزش‌صدق فرمول $A$ برابر $i$ و ارزش‌صدق فرمول $B$ برابر $j$ باشد، ارزش‌صدق $A\to B$ را بزرگترین عدد حقیقی $r\leq 1$ تعریف می‌کنیم که
$\text{min}\{r,i\}\leq j$.
جدول ارزش $\to$ را برای هر نه حالت ممکن از ارزش‌صدق طرفین آن رسم کنید.
\item
مقادیری از $p$ و $q$ را بیابید که برای آن‌ها ارزش‌صدق $(((p\to q)\to p)\to p)$ برابر $1$ نیست.
\item
نشان بدهید هر استنتاج $D$ که تنها از قانون
\lr{(Axiom)}
و قوانین معرفی و حذف $\to$ استفاده کند، دارای این ویژگی است: اگر $A$ نتیجه‌ی $D$ باشد و $v$ ارزیابی باشد که $v(A)<1$، مقدمه‌ی حذف‌نشده‌ی $B$ای در $D$ هست چنانکه $v(B)\leq v(A)$.
\item
با استفاده از نتایج به‌دست‌آمده در بخش‌های قبل ثابت کنید قانون پرس را نمی‌توان تنها با استفاده از قانون
\lr{(Axiom)}
و قوانین معرفی و حذف $\to$ ثابت کرد.
\end{enumerate}

\pagebreak

\item
فرض کنید
$A(p_1,\ldots,p_n)$
یک فرمول و $v$ یک ارزیاب باشد. فرمول‌های $q_1$، \ldots، $q_n$ و فرمول $B$ را چنین تعریف می‌کنیم:
\begin{itemize}[label={--}]
\item
برای هر $1\leq i\leq n$، اگر $v(p_i)=T$ آنگاه $q_i=p_i$ و در غیر این صورت $q_i=\neg q_i$.

\item
اگر $v(A)=T$ آنگاه $B=A$ و در غیر این صورت $B=\neg A$.
\end{itemize}
ثابت کنید
$\{q_1,\ldots,q_n\}\vdash B$.

\item
مجموعه‌ی نامتناهی
$\{A_1,A_2,\ldots\}$
را در نظر بگیرید. فرض کنید برای هر ارزیاب $v$ عدد $n$ وجود دارد چنانکه $v(A_n)=T$. ثابت کنید عدد $m$ وجود دارد که
$\models A_1\vee\ldots\vee A_m$.

\item
فرض کنید $\Gamma$ مجموعه‌ای از فرمول‌ها و $A$ یک فرمول باشد. فرض کنید $\Gamma\cup \{A\}$ حاوی یک زیرمجموعه‌ی متناهی ارضاناپذیر و $\Gamma\cup\{\neg A\}$ نیز حاوی یک زیرمجموعه‌ی متناهی ارضاناپذیر باشد. ثابت کنید $\Gamma$ حاوی یک زیرمجموعه‌ی متناهی ارضاناپذیر است.

\item
فرض کنید $\Gamma$ و $\Sigma$ دو مجموعه از فرمول‌ها باشند چنانکه $\Gamma\neq\emptyset$ و
$\Sigma\cup\{\neg A|A\in\Gamma\}$
ناسازگار است. نشان دهید عدد طبیعی $n$ و
$$
A_1,\ldots,A_n\in\Gamma
$$
چنان موجود است که
$\Sigma\vdash A_1\vee\ldots\vee A_n$.

\item
برای هر فرمول $A$ و اتم $p$ فرض کنید
$A^*=A[\top/p]\vee A[\bot/p]$.
ثابت کنید:
\begin{enumerate}
\item
$A\models A^*$
\item
اگر
$A\models B$
و $p$ در $B$ موجود نباشد، آنگاه
$A^*\models B$.
\item
اگر
$A\models B$
آنگاه فرمولی مانند $C$ موجود است که اتم‌های آن مشترک میان $A$ و $B$ است و $A\models C$ و $C\models B$.
\end{enumerate}

\item
در فرمول $A$ به جای هر وقوع اتم $p$ فرمول $\neg p$ را جایگزین می‌کنیم و حاصل را $A'$ می‌نامیم. ثابت کنید $A$ توتولوژی است اگر و فقط اگر $A'$ توتولوژی است.

\end{enumerate}

\end{document}
